\documentclass[hyperref,bachelorofscience]{cgvpub}
%weitere Optionen zum Ergänzen (in eckigen Klammern):
% 
% bibnum	numerische Literaturschlüssel
% final 	für Abgabe	
% lof			Abbildungsverzeichis
% lot			Tabellenverzeichnis
% noproblem	keine Aufgabenstellung
% notoc			kein Inhaltsverzeichnis
% twoside		zweiseitig
\author{Roland Raytracer}
\title{Titeltext}
\birthday{1. Januar 1234}
\placeofbirth{Dresden}
\matno{123456}
\betreuer{Dr. B. Spline}
\bibfiles{literatur}
\problem{Text der Aufgabenstellung...}
\copyrighterklaerung{Hier soll jeder Autor die von ihm eingeholten
Zustimmungen der Copyright-Besitzer angeben bzw. die in Web Press
Rooms angegebenen generellen Konditionen seiner Text- und
Bild"ubernahmen zitieren.}
\acknowledgments{Die Danksagung...}
\abstracten{abstract text english}
\abstractde{ Zusammenfassung Text Deutsch}
\usepackage{rotating}

\begin{document}
\chapter{Introduction/Preliminaries}
Motivation
\cite{test}
\section{Wikidata}
Wikidata is the knowledge base sister project of Wikipedia. It is a public, open database system that, at its core, stores statements about specific items. The basic unit of information is a triple subject-predicate-object stating an item (subject) has a property (predicate) which is either another item or a value of some kind (object). 
\chapter{Property Constraints}
To improve the data quality, help editors avoid common mistakes and clarify the usage of properties, Wikidata allows constraints to be defined on properties. They are however not firm rules and execptions can and should be made if necessary. A property constraint says something about the way a property may or may not be used. This can mean simple restrictions on the statement a property is part of, e.g. only certain values are allowed or more complex requirements on connected parts of the data graph, e.g. the value of a statement with this property should have certain statements. On the following page you can find table \ref{tab_basestats} which list the existing property constraints and a short description as well as classifying information which will be explained shortly. Afterwards the constraints will be explained in detail and either a basic approach for their modeling or reasoning why it is not possible or sensible to model it given.\\

To apply a constraint to a property a statement <property> <P2302> <constraint> is added to the properties item page, where <property> is the property that should be constrained, <P2302> (property constraint) denotes that the property has a constraint and <constraint> is the specific constraint. Note that a constrained statement from hereon denotes a statement whose predicate is a property that has a specific constraint. A simple example for this would be distinct values (Q21502410) stating that values of the property must be unique across all constrained statements. For this case no additional information regarding the constraint is necessary and violation could be found by surveying all constrained statements and comparing their values.\\
Constraints can be further specified by using qualifiers on the constraint statement. These qualifiers are either optional or mandatory and may be applied once or multiple times. For example, the none of constraint (Q52558054) states that values for a specific property may not be in a set of disallowed values. These are specified as qualifiers using item of property constraint (P2305) on the constraint statement and require at least one such statement.\\
More comlex dependencies are also possible. A constraint can depend on all constrained statements of a specific item (single value constraint [Q19474404]), on all statements of the value of a constrained statement (value requires statement constraint [Q21510864]) or in the most complex case on all statements using the subclassOf-predicate (type constraint [Q21503250]).
\newpage
\begin{table}[t]
\caption{property constraints and basic information}\label{tab_basestats}
\begin{tabular}{rcl}

\# of properties & constraint id & constraint name \\
\hline
4265 & Q21503247 & Item requires statement \\
3021 & Q21502404 & Format \\
2909 & Q19474404 & Single value \\
2843 & Q21502410 & Unique value \\
2613 & Q21503250 & Type \\
1410 & Q53869507 & Scope \\
781 & Q21502838 & Conflicts with \\
707 & Q21510865 & Value type \\
343 & Q21514353 & Allowed units \\
303 & Q21510851 & Allowed qualifiers \\
244 & Q21510864 & Value requires statement \\
203 & Q21510856 & Mandatory qualifier \\
188 & Q21510860 & Range \\
119 & Q21510855 & Inverse \\
116 & Q21510859 & One of \\
106 & Q52848401 & Integer \\
65 & Q51723761 & No bounds \\
53 & Q21510852 & Commons link \\
33 & Q21510862 & Symmetric \\
27 & Q52004125 & Allowed entity types \\
17 & Q52558054 & None of \\
13 & Q21510857 & Multi-value \\
10 & Q54554025 & Link \\
6 & Q21510854 & Difference within range \\
3 & Q52060874 & Single best value \\
1 & Q25796498 &  \\
1 & Q52712340 &  \\

\end{tabular}
\end{table}
\newpage



\chapter{Rules}
\chapter{Evaluation/Comparison}
\end{document}