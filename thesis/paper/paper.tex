\documentclass[hyperref,bachelorofscience]{cgvpub}
%weitere Optionen zum Ergänzen (in eckigen Klammern):
% 
% bibnum	numerische Literaturschlüssel
% final 	für Abgabe	
% lof			Abbildungsverzeichis
% lot			Tabellenverzeichnis
% noproblem	keine Aufgabenstellung
% notoc			kein Inhaltsverzeichnis
% twoside		zweiseitig
\author{Roland Raytracer}
\title{Titeltext}
\birthday{1. Januar 1234}
\placeofbirth{Dresden}
\matno{123456}
\betreuer{Dr. B. Spline}
\bibfiles{literatur}
\problem{Text der Aufgabenstellung...}
\copyrighterklaerung{Hier soll jeder Autor die von ihm eingeholten
Zustimmungen der Copyright-Besitzer angeben bzw. die in Web Press
Rooms angegebenen generellen Konditionen seiner Text- und
Bild"ubernahmen zitieren.}
\acknowledgments{Die Danksagung...}
\abstracten{abstract text english}
\abstractde{ Zusammenfassung Text Deutsch}

\usepackage{rotating}
\usepackage{float}
\restylefloat{table}
\usepackage{makecell}
\renewcommand{\cellalign}{tl}
\usepackage{tabularx}

\begin{document}
\chapter{Introduction/Preliminaries}
Motivation
\cite{test}
\section{Wikidata}
Wikidata is the knowledge base sister project of Wikipedia. It is a public, open database system that, at its core, stores statements about specific items. The basic unit of information is a triple subject-predicate-object stating an item (subject) has a property (predicate) which is either another item or a value of some kind (object). These store the relations of entities with each other or with specific data. An entity is either an item, which represents a topic, concept or real world entity, or a property, which denotes a relation an item has to another or with a value. A value is one of several datatypes noted in table \ref{tab_datatypes}. \\

\begin{table}[H]
\caption{wikidata datatypes}\label{tab_datatypes}
\begin{tabularx}{\textwidth}{lllX}
Datatype & Fields & Description \\
\hline
string & string (string) & simple string literal \\
globecoordinate & \makecell{latitude (float) \\ longitude (float) \\ precision (float) \\ globe(URI)} & \makecell{coordinates on a given celestial body \\ or geographic standard} \\
quantity & \makecell {amount (decimal) \\ upperBound (decimal) \\ lowerBound (decimal) \\ unit (URI or 1)} & quantity of a specified unit (or none) \\ with uncertainty \\
time & \makecell {time (string) \\ calendarmodel (URI) \\ precision (integer)} & \makecell {time as a string according to calendarmodel \\ (julian or gregorian),\\ with precision mapping to orders of magnitude \\ (eg. 10 years, months)} \\


\end{tabularx}
\end{table}

While this is simliar to RDF and Wikidata has in fact been exported as RDF \cite{EGKMV2014}, the data structure is more complex than simple triples. It allows, in essence, to store additional information about statements. This is done via a statement id, which is assigend to every triple storing information about an item. 
\section{Datalog}
Datalog is a logic programming language. It is fully delarative and uses simple rules to derive additional knowledge from given facts. The most basic unit of information in datalog is a constant, which is a given [string]? expressing a fix point. In the context of Wikidata any item or property, e.g. the universe (Q1) or head of government (P6), could be represented as a constant.
\section{Representation}
To model the basic represenation of triples, qualifiers and references three predicates are introduced.\\
tripleEDB (STATEMENT, ITEM, PROPERTY, VALUE) represents a single statement consisting of the statement ID, the item ID, the property ID and the value as either and item ID or a string-representation of the data value.\\
qualifierEDB (STATEMENT, PROPERTY, VALUE) and referenceEDB (STATEMENT, PROPERTY, VALUE) represent the qualifiers and references on a given statement ID with their respective property id and the value as usual.\\

Additional predicates modelling different preprocessing steps are explained in chapter \ref{cha_rules}.
\section{Terminology}
\begin{table}[H]
\caption{shorthands}
\begin{tabular}{ll}
Shorthand & Description \\
\hline
constrained statement & statement with the constrained property as predicate \\
constrained qualifier & qualifier with the constrained property as predicate \\
constrained reference & reference with the constrained property as predicate \\
constrained item & item with a constrained statement \\
item statement & statement on the same item as the constrained statement \\
\end{tabular}
\end{table}
\chapter{Property Constraints}

%\begin{table}[t]
%\caption{property constraints and basic information}\label{tab_basestats}
%\begin{tabular}{rcl}

%\# of properties & constraint id & constraint name \\
%\hline
%4265 & Q21503247 & Item requires statement \\
%3021 & Q21502404 & Format \\
%2909 & Q19474404 & Single value \\
%2843 & Q21502410 & Unique value \\
%2613 & Q21503250 & Type \\
%1410 & Q53869507 & Scope \\
%781 & Q21502838 & Conflicts with \\
%707 & Q21510865 & Value type \\
%343 & Q21514353 & Allowed units \\
%303 & Q21510851 & Allowed qualifiers \\
%244 & Q21510864 & Value requires statement \\
%203 & Q21510856 & Mandatory qualifier \\
%188 & Q21510860 & Range \\
%119 & Q21510855 & Inverse \\
%116 & Q21510859 & One of \\
%106 & Q52848401 & Integer \\
%65 & Q51723761 & No bounds \\
%53 & Q21510852 & Commons link \\
%33 & Q21510862 & Symmetric \\
%27 & Q52004125 & Allowed entity types \\
%17 & Q52558054 & None of \\
%13 & Q21510857 & Multi-value \\
%10 & Q54554025 & Link \\
%6 & Q21510854 & Difference within range \\
%3 & Q52060874 & Single best value \\
%1 & Q25796498 &  \\
%1 & Q52712340 &  \\
%
%\end{tabular}
%\end{table}
To improve the data quality, help editors avoid common mistakes and clarify the usage of properties, Wikidata allows constraints to be defined on properties. They are however not firm rules and execptions can and should be made if necessary. A property constraint says something about the way a property may or may not be used. This can mean simple restrictions on the statement a property is part of, e.g. only certain values are allowed or more complex requirements on connected parts of the data graph, e.g. the value of a statement with this property should have certain statements. On the following page you can find table \ref{tab_basestats} which list the existing property constraints and a short description as well as classifying information which will be explained shortly. Afterwards the constraints will be explained in detail and either a basic approach for their modelling or reasoning why it is not possible or sensible to model it given.\\

To apply a constraint to a property a statement <property> <P2302> <constraint> is added to the properties item page, where <property> is the property that should be constrained, <P2302> (property constraint) denotes that the property has a constraint and <constraint> is the specific constraint. Note that a constrained statement from hereon denotes a statement whose predicate is a property that has a specific constraint. A simple example for this would be distinct values (Q21502410) stating that values of the property must be unique across all constrained statements. For this case no additional information regarding the constraint is necessary and violation could be found by surveying all constrained statements and comparing their values.\\
Constraints can be further specified by using qualifiers on the constraint statement. These qualifiers are either optional or mandatory and may be applied once or multiple times. For example, the none of constraint (Q52558054) states that values for a specific property may not be in a set of disallowed values. These are specified as qualifiers using item of property constraint (P2305) on the constraint statement and require at least one such statement.\\
More comlex dependencies are also possible. A constraint can depend on all constrained statements of a specific item (single value constraint [Q19474404]), on all statements of the value of a constrained statement (value requires statement constraint [Q21510864]) or in the most complex case on all statements using the subclassOf-predicate (type constraint [Q21503250]).


\section{Wikidata constraints explained}
This section will explain all interesting, meaning translatable or partially translatable, currently used property constraints on Wikidata. They will be presented rougly in descending order of the number of properties with this constraint, although constraints with similar function will be explained directly after one another. Also, all constraints that were considered untranslatable are listed at the and, each with a short explanation as to why they were placed in that group.\\

Every constraint will start with a table showing basic properties of the constraint. This includes the requirement, relevant qualifiers, possible violations of the constraint and the data the constraint depends on. The requirement states which part of the data graph in relation to the constrained statement are restricted or need to exist. The qualifiers list their cardinality, meaning if they are required and if they may be used multiple times, and a short description. The violations list the conflicting (part of a) statement in relation to the constrained statement. The dependecies list their relation to the constrained statement. Table \ref{tab_example_constraint} shows the formatting as an example.

\begin{table}[H]
\caption{example constraint (Q31415926)}\label{tab_example_constraint}
\begin{tabularx}{\textwidth}{ ll X}
\hline
Requirement & requirement \\
\hline
Qualifiers & \makecell{example qualifier (P1234) -- 1..* \\ description \\ another qualifier (P4321) -- 0..1 \\ different description} \\
\hline
Violation & \makecell{one way to violate the constraint \\ another violation possibility} \\
\hline
Dependency & statements, qualifiers or references in relation to the constrained statement\\
\hline
Rule features & features needed (missing if no features needed) \\
\hline
\end{tabularx}
\end{table}

\newpage
\subsection{Item requires statement (Q21503247)}\label{subsec_item_requires_statement}
The item requires statement constraint requires an additional statement on an item with a statement using the constrained property, possibly also using a specific value from the list of allowed values.
\begin{table}[H]
\caption{item requires statement constraint (Q21503247)}
\begin{tabularx}{\textwidth}{ ll X}
\hline
Requirement & item statement \\
\hline
Qualifiers & \makecell{required property (P2306) -- 1 \\ property the item statement must have \\ allowed value (P2305) -- 0..* \\ values the item statement may have} \\
\hline
Violation & \makecell{no item statement with required property \\ all item statements with required property have no allowed value} \\
\hline
Dependency &  all item statements\\
\hline
Rule features & negation \\
\hline
\end{tabularx}
\end{table}

An item violates the constraint if it a) has a statement using constrained property as predicate, b1) has no statement using the required property as predicate, or (if allowed values are specified) b2) has only statement using required property as predicate that do not use one of the allowed values as value.\\
While condition a) can easily be tested by a tripleEDB-atom with the constrained property, condition b1) and b2) pose the question if an item does not have a statement. This is adressed in section [?].

\subsection{Value requires statement (Q21510864)}\label{subsec_value_requires_statement}
The value requires statement constraint requires an additional statement on the value of constrained statement, possibly also using a specific value from the list of allowed values. Since the constraint works very similar to item requires statement (see \ref{subsec_item_requires_statement}) --- requiring a statement on the value of the constrained statement instead of on the item --- it is not elaborated on.

\subsection{Inverse (Q21510855)}\label{subsec_inverse}
The inverse constraint requires an additional statement on the value of a constrained statement with a specific property and the constrained item as value. It is very similar to value requires statement (see \ref{subsec_value_requires_statement}), although it does not specify allowed values, since the required statement has one specific value: the constrained item. While this means it could not be modelled exactly as a value requires statement constraint, it would only require very small rule changes and will thus not be elaborated on.

\subsection{Symmetric (Q21510862)}
The symmetric constraint requires an additional statement on the value of a constrained statement with the constrained property and the constrained item as value. It is thus equal to an inverse constraint (see \ref{subsec_inverse}) with the same property as constrained property and required property and will not be elaborated on.

\subsection{Single value (Q19474404)}
The single value constraint constrains the values a property may take on a single item. A property marked with the single value constraint can have properties marked as separators, which define exceptions under specific circumstances.
\begin{table}[H]
\caption{single value constraint (Q19474404)}
\begin{tabularx}{\textwidth}{ ll X}
\hline
Requirement & constrained statement value \\
\hline
Qualifiers & \makecell{separator (P4155) -- 0..* \\ qualifiers that must be different for exception} \\
\hline
Violation & \makecell{two constrained statements with same item and value \\ (exception: statements with separators as qualifiers with different values)} \\
\hline
Dependency & \makecell{without separators: all constrained statements on one item \\ with separators: additionally all qualifiers with separators on those statements }\\
\hline
Rule features & inequality, partially negation \\
\hline
\end{tabularx}
\end{table}

If no separators are specified, two triple form a violation of this constraint if they a) belong to the same item, b) have the same constrained property, and c) have the same value.
If separators are specified, a violation as above can be ignored if the statements each have at least one qualifier with the same separator as predicate and different values.\\
In the case without separators the modelling idea would be to find two triples tripleEDB with the same item and constrained property and test if their id is unequal. If separators were specified a violation would be the two triples and all their separator qualifiers either equal or nonexistent.

\subsection{Distinct values (Q21502410)}
The distinct values constraint forbids the existence of two constrained statements with the same value.
\begin{table}[H]
\caption{distinct values constraint (Q21502410)}
\begin{tabularx}{\textwidth}{ ll X}
\hline
Requirement & constrained statement value \\
\hline
Violations & two constrained statements with same value \\
\hline
Dependency & all constrained statements\\
\hline
Rule features & inequality \\
\hline
\end{tabularx}
\end{table}
To model the distinct values constraint it is necessary to find triples tripleEDB with the constrained property and the same value whose statement id is unequal.

\subsection{Type (Q21503250)}\label{subsec_type}
The type constraint requires the existence of either one of a specific statement (instanceOf <class>) or a chain of statements using subclassOf to connect directly (subclassOf* <class>)
or via instanceOf (instanceOf/subclassOf* <class>) for all classes on the constrained item.
\begin{table}[H]
\caption{type constraint (Q21503250)}
\begin{tabularx}{\textwidth}{ ll X}
\hline
Requirement & chain of statements \\
\hline
Qualifiers & \makecell{relation (P2309) -- 1 \\ must be instance of (Q21503252), subclass of (Q21514624), \\ or instance or subclass of (Q30208840); \\ specifies the relation the constrained item must have to the classes\\
class (P308) -- 1..* \\ entities that must be reached using the relation} \\
\hline
Violation & constrained item without relation to a class \\
\hline
Dependency & \makecell{statements of constrained items using instanceOf (P31) or subclassOf(P279) \\ all statements using subclassOf (P279) }\\
\hline
Rule features & inequality, negation \\
\hline
\end{tabularx}
\end{table}

If the relation is instanceOf, the constraint can be evaluated like multiple item requires statement constraints (see \ref{subsec_item_requires_statement}) by setting the required property to instanceOf (P31) and, for each created item requires statement constraint, the allowed value to one of the classes. \\
For the relations involving subclasses all transivite subclass relationships would be concluded by setting a subclass relationship if a third item can be found that is subclass of one and subclassed by the other item. This would be extended with one step of instanceOf. A violation would then be any constrained item that is not subclass of any of the classes.

\subsection{Value type (Q21510865)}
The value type constraint works like the type constraint (see \ref{subsec_type}) except the statement chains are required on the value of the constrained statement instead of the constrained item. It will thus not be expanded upon.

\subsection{Property scope (Q53869507)}
The property scope constraint specifies if a property may be used in a statement, qualifier or reference as predicate.
\begin{table}[H]
\caption{property scope constraint (Q53869507)}
\begin{tabularx}{\textwidth}{ ll X}
\hline
Requirement & constrained statement \\
\hline
Qualifiers & \makecell{property scope (P5314) -- 1..3 \\ must be as main value (Q54828448), as qualifier (Q54828449) or as reference (Q54828450) \\ specifies allowed use in statements, qualifiers or references} \\
\hline
Violation & constrained statement in a position that is not allowed \\
\hline
Dependency & constrained statement, qualifier or reference \\
\hline
\end{tabularx}
\end{table}

Since only three posibilities exist, this is equivalent to forbidding all other possibilities and the rules are simple mapping all of tripleEDB, qualifierEDB and referenceEDB with the constrained property that are not allowed.

\subsection{Conflicts-with (Q21502838)}
The conflicts with constraint forbids either certain statements or statements with a specific property on constrained items.
\begin{table}[H]
\caption{Conflicts-with constraint (Q21502838)}
\begin{tabularx}{\textwidth}{ ll X}
\hline
Requirement & item statement \\
\hline
Qualifiers & \makecell{conflicting property (P2306) -- 1 \\ property no item statement may have (with conflicting values if specified) \\
conflicting values (P2305) -- 0..* \\ values that may not appear in item statements with the conflicting property} \\
\hline
Violation & \makecell{item statement with the conflicting property \\ item statement with the conflicting property and one of the conflicting values} \\
\hline
Dependency & all item statements \\
\hline
\end{tabularx}
\end{table}
The rule modelling can simply be done by choosing all tripleEDB with the constrained property and then all tripleEDB on the same item with the conflicting property, specifying the values as well if necessary.



\chapter{Rules}\label{cha_rules}
\chapter{Evaluation/Comparison}
\end{document}
